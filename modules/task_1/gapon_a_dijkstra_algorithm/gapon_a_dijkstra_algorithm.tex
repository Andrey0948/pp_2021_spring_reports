\documentclass{report}

\usepackage[T2A]{fontenc}
\usepackage[utf8]{luainputenc}
\usepackage[english, russian]{babel}
\usepackage[pdftex]{hyperref}
\usepackage[14pt]{extsizes}
\usepackage{listings}
\usepackage{color}
\usepackage{geometry}
\usepackage{enumitem}
\usepackage{multirow}
\usepackage{graphicx}
\usepackage{indentfirst}
\usepackage[ruled,vlined,linesnumbered]{algorithm2e}

\geometry{a4paper,top=2cm,bottom=3cm,left=2cm,right=1.5cm}
\setlength{\parskip}{0.5cm}
\setlist{nolistsep, itemsep=0.3cm,parsep=0pt}

\newenvironment{pseudocode}[1][htb]
  {\renewcommand{\algorithmcfname}{Алгоритм}
   \begin{algorithm}[#1]%
  }{\end{algorithm}}


\lstset{language=C++,
		basicstyle=\footnotesize,
		keywordstyle=\color{blue}\ttfamily,
		stringstyle=\color{red}\ttfamily,
		commentstyle=\color{green}\ttfamily,
		morecomment=[l][\color{magenta}]{\#}, 
		tabsize=4,
		breaklines=true,
  		breakatwhitespace=true,
  		title=\lstname,       
}

\makeatletter
\renewcommand\@biblabel[1]{#1.\hfil}
\makeatother

\begin{document}

\begin{titlepage}

\begin{center}
Министерство науки и высшего образования Российской Федерации
\end{center}

\begin{center}
Федеральное государственное автономное образовательное учреждение высшего образования \\
Национальный исследовательский Нижегородский государственный университет им. Н.И. Лобачевского
\end{center}

\begin{center}
Институт информационных технологий, математики и механики
\end{center}

\vspace{4em}

\begin{center}
\textbf{\LargeОтчет по лабораторной работе} \\
\end{center}
\begin{center}
\textbf{\LargeПоиск кратчайших путей из одной вершины. Алгоритм Дейкстры} \\
\end{center}

\vspace{4em}

\newbox{\lbox}
\savebox{\lbox}{\hbox{text}}
\newlength{\maxl}
\setlength{\maxl}{\wd\lbox}
\hfill\parbox{7cm}{
\hspace*{5cm}\hspace*{-5cm}\textbf{Выполнил:} \\ студент группы 381806-3 \\ Гапон А.К.\\
\\
\hspace*{5cm}\hspace*{-5cm}\textbf{Проверил:}\\ доцент кафедры МОСТ, \\ кандидат технических наук \\ Сысоев А. В.
}

\vspace{\fill}

\begin{center} Нижний Новгород \\ 2021 \end{center}

\end{titlepage}

\setcounter{page}{2}

\tableofcontents
\newpage

\section*{Введение}
\addcontentsline{toc}{section}{Введение}
Алгорим дейкстры (англ. Dijkstra’s algorithm) — алгоритм на графах, изобретённый нидерландским учёным Эдсгером Дейкстрой в 1959 году. Находит кратчайшие пути от одной из вершин графа до всех остальных. Алгоритм работает только для графов без рёбер отрицательного веса. Алгоритм широко применяется в программировании и технологиях, например, его используют протоколы маршрутизации OSPF и IS-IS.
\par Шаг алгоритма.
Если все вершины посещены, алгоритм завершается.
В противном случае, из ещё не посещённых вершин выбирается вершина u, имеющая минимальную метку.
Мы рассматриваем всевозможные маршруты, в которых u является предпоследним пунктом. Вершины, в которые ведут рёбра из u, назовём соседями этой вершины. Для каждого соседа вершины u, кроме отмеченных как посещённые, рассмотрим новую длину пути, равную сумме значений текущей метки u и длины ребра, соединяющего u с этим соседом.
Если полученное значение длины меньше значения метки соседа, заменим значение метки полученным значением длины. Рассмотрев всех соседей, пометим вершину u как посещённую и повторим шаг алгоритма.
\par В данной лабораторной работе будет рассмотрен алгоритм Дейкстры для нахождения крайтчайшего пути в графе от заданной начальной и конечной вершины с использованием технологий параллельных вычислений.
\newpage

\section*{Постановка задачи}
\addcontentsline{toc}{section}{Постановка задачи}
В данной лабораторной работе необходимо разработать несколько функций с поддержкой различных технологий параллелизма, реализующих алгоритм Дейкстры для нахождения кратчайшего пути.
\par В лабораторных работах реализовать:
\begin{itemize}
\item последовательный метод алгоритма Дейкстры;
\item параллельный метод алгоритма Дейкстры с помощью технологии TBB;
\item параллельный метод алгоритма Дейкстры с помощью технологии OpenMP;

\end{itemize}


\newpage

\section*{Схема распараллеливания}
\addcontentsline{toc}{section}{Схема распараллеливания}
В алгоритме Дейкстры можно распараллелить два основных момента внутри главного цикла, который обходит граф по всем вершинам:
\begin{itemize}
\item изменение расстрояния до следующей вершины
\item поиск вершины с минимальной меткой
\end{itemize}
\newpage

\section*{Описание программной реализации}
\addcontentsline{toc}{section}{Описание программной реализации}
Вектор proces – хранение информации o посещенных вершинах.
Веток min dist, хранящий найденные кратчайшие пути. 
Изначально все вершиныы помечаем, как не посещенныя, т. е. элементам вектора proces присвoенo значение false. В вектор min dist записываем  максимальное возможное значение int.
\begin{lstlisting}
 std::vector<int> min dist(size, max_weight);
 std::vector<bool> proces(size, false);
 std::vector<int> res;
\end{lstlisting}
\par Create Graph - функция для генерациии случайного графа.
\begin{lstlisting}
std::vector<int> Create_Graph(int size)
\end{lstlisting}

\par Входные параметры: граф, номер начальной и конечной вершины, для которых ищем кратчайший путь.

\subsection*{OpenMP}
\addcontentsline{toc}{subsection}{OpenMP}
с помощью дириктивые ниже, создается распараллеливание алгоритма
\par \verb|#pragma omp parallel firstprivate(min_weight, min_index)| 
\par Внутри данной параллельной секции с помощью директивы: 
\par\verb|#pragma omp parallel for | 
\par распараллеливается цикл нахождения вершины с наименьшим расстоянием до заданной начальной вершины. Переменные \verb| min_weight, min_index | с помощью параметра \verb|firstprivate| создаются локальными для каждого потока и инициализируются значениями исходных переменных. Таким образом, каждый поток находит свою локальную вершину с минимальным расстоянием. Для того, чтобы найти среди всех вершин глобальную вершину с минимальным расстоянием создается критическая секция с помощью директивы \verb|#pragma omp critical|. В данную критичесую секцию каждый поток по очереди заходит, сравнивает свою локальную вершину с глобальной и если локальная вершина имеет расстояние меньшее, чем у глобальной вершины, то локальная вершина становится глобальной. В конце будет найдена общая вершина с наименьшим расстоянием до начальной.
\par Для распараллеливания цикла, в котором перезаписывались расстояния от начальной до всех непройденных вершин, смежных с найденной использовалась директива \verb|#pragma omp parallel for|.

\subsection*{TBB}
\addcontentsline{toc}{subsection}{TBB}
В версии TBB используется специальный инструмент:
\par\verb|tbb::parallel_reduce(...)|
\par с помощью которого можно распараллелить нахождение вершины с наименьшим расстоянием. На вход подается: \par\verb|tbb::blocked_range<int>(0, size, grainsize)| 
\par итерационное пространство с указанными границами [0; size), где size~--- количество вершин графа, глобальная вершину, которая имеет изначально максимальное расстояние до начальной вершины и лямбда-функция \par\verb|[&](const tbb::blocked_range<int>& range, Min local) | 
\par которая производит нахождение локального минимума (Min - это структура, содержащая вес и номер вершины) 
\par Также нужно передать еще одну лямбда-функцию, которая производит редукцию наших данных, а именно сравнивает между собой вершины по расстоянию.
\par Для распараллеливания цикла, в котором перезаписывались расстояния от начальной вершины до всех непройденных смежных вершин, использовался инструмент  \verb|tbb::parallel_for(...)|. На вход подается одномерное итерационное пространство \verb|tbb::blocked_range|, а также лямбда-функция, производящая необходимые вычисления.
\newpage


\section*{Подтверждение корректности}
\addcontentsline{toc}{section}{Подтверждение корректности}
Для подтверждения правильности в программе представлен набор тестов.
\par Тесты проверяют корректность работы параллельных версий алгоритма Дейкстры и сравненивают с последовательной версией для случайных графов.
\newpage

\section*{Результаты экспериментов}
\addcontentsline{toc}{section}{Результаты экспериментов}
Вычислительные эксперименты для оценки эффективности алгоритма Дейкстра для нахождения крайтшайших путей в графе проводились на оборудовании со следующей аппаратной конфигурацией:

\begin{itemize}
\item Процессор: Intel Core i5-3230М, 2 ядра;
\item Оперативная память: 6 GB;
\item ОС: Microsoft Windows 10.
\end{itemize}

\par Эксперименты проводились на графе с \verb|10 000| вершинами. 
\par Результаты экспериментов:
\begin{table}[!h]
\centering
\begin{tabular}{lllll}
Кол-во потоков & Пoследоватeльно,с & OpenMP,с & Ускорение  \\
2 & 0.919297 & 0.556642 & 1.62  \\
4 & 0.919297 & 0.391076 & 2.34  \\
\end{tabular}
\caption{Результаты вычислительных экспериментов OpenMP}
\end{table}

\begin{table}[!h]
\centering
\begin{tabular}{lllll}
Кол-во потоков & Последовательно,с & TBB,с & Ускорение  \\
2 & 0.919297 & 0.559033 & 1.63  \\
4 & 0.919297 & 0.423855 & 2.18  \\
\end{tabular}
\caption{Результаты вычислительных экспериментов TBB}
\end{table}


\par В результате тестов видна эффективность параллельных методов. OpenMP показывает лучший результат, чем TBB. следовательно TBB не очень подходит для алгоритмов, связанных с графами.
\newpage


\section*{Заключение}
\addcontentsline{toc}{section}{Заключение}
В результате лабораторной рассматривали последовательный и параллельные методы работы алгоритма Дейкстры .
Параллельные версии действительно работают эффективнее, чем последовательная, о чем говорят результаты экспериментов, проведенных в ходе работы.
Проведя анализ, наиболее эффективной оказалась технология OpenMP, менее эффективной технология TBB.

\newpage
\begin{thebibliography}{1}
\addcontentsline{toc}{section}{Список литературы}
\bibitem{Wiki1} Wikipedia: the free encyclopedia [Электронный ресурс] // URL: https://en.wikipedia.org/wiki/
\bibitem{Sysoev} Сысоев А.В., Мееров И.Б., Свистунов А.Н., Курылев А.Л., Сенин А.В., Шишков А.В., Корняков К.В., Сиднев А.А. «Параллельное программирование в системах с общей памятью. Инструментальная поддержка». Учебно-методические материалы по программе повышения квалификации «Технологии высокопроизводительных вычислений для обеспечения учебного процесса и научных исследований». Нижний Новгород, 2007, 110 с. 
\end{thebibliography}
\newpage

\end{document}